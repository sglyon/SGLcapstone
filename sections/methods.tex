%!TEX root = ../myStyle/Capstone.tex

\section{Methods} \label{sec:methods}

In this section I describe the different approaches that were employed and considered during the creation of \texttt{hsfpy}. I will give an overview of the tools that were considered for this project as well as a short usage example for each tool. To maintain consistency and make differences across the methods more apparent, I will use a selection of the code from the \texttt{hsf} C++ library. The main components of this example code are a C++ class \texttt{HKnotVector}, a function \texttt{numClamp}, and a few typedefs, \texttt{DoubleVec, IntVec}, and \texttt{IntVecVec}. The goal for each example will be to correctly wrap the \texttt{HKnotVector} class, as it uses the other components. The source code can be found in Appendix~\ref{app:code}.

\subsection{SWIG} \label{sub:swig}

  SWIG\footnote{SWIG is free and open source. The source code is hosted at \url{https://github.com/swig/swig} and the homepage for the project is \url{http://www.swig.org/}.} is an acronym meaning simplified wrapper and interface generator. The following excerpt from the SWIG homepage provides a good explanation of what SWIG is commonly used for:

  \setstretch{1.0}
  \begin{quote}
    SWIG is a software development tool that connects programs written in C and C++ with a variety of high-level programming languages. SWIG is used with different types of target languages including common scripting languages such as Perl, PHP, Python, Tcl and Ruby\ldots SWIG is most commonly used to create high-level interpreted or compiled programming environments, user interfaces, and as a tool for testing and prototyping C/C++ software. SWIG is typically used to parse C/C++ interfaces and generate the 'glue code' required for the above target languages to call into the C/C++ code.
  \end{quote}
  \mainstretch{}

  \noindent  SWIG is a very well-established project; the first version appeared in July 1995 and the most recent version was released in May 2013. Over the years, SWIG has developed in to a very powerful and flexible tool. The best expression of this flexibility is that SWIG officially has at least partial support for nineteen different target languages, whereas other tools that will be discussed in this section are Python specific. A great aspect of this flexibility is that users can run SWIG on the exact same set of files and generate wrappers for different target languages by simply changing a single command line argument.

  However, SWIG is not a perfect tool. Due in part to the freedom it gives users to choose amongst multiple output languages, SWIG generates wrapper code that is relatively difficult to customize for a specific target language. Due to its multiple output paradigm, SWIG doesn't fully support all C++ language features for every output language. Furthermore, in order to use SWIG, a user must supply an additional interface file (commonly with a \texttt{.i} suffix) in which the user uses a C-like syntax to describe the desired interface. Finally, the last main drawback I noticed when testing SWIG for \texttt{hsfpy} is that the building/compiling phase for SWIG is non-trivial.  \REVIEW{You should mention that hsf uses a lot of the C++ functionality that SWIG doesn't support for python (and don't use a lot :))}

  \subsubsection{SWIG Usage Example} \label{ssub:swig_usage_example}

    To give an idea of how to use SWIG, I outline how to construct a Python interface to the code contained in Appendix~\ref{app:code}. The first step is to create a SWIG interface file where the desired wrapper is designed. I will present the wrapper used to expose the class \texttt{HKnotVector}, and then explain the key components.

    \setstretch{0.7}
    \vspace{.2in}
    \lstinputlisting[language=C++,label=code:HKVswig, caption=\texttt{HKnotVector.i}: SWIG interface for HKnotVector]{../code/swig/HKnotVector.i}
    \setstretch{1.35}

    \begin{itemize}
      \item \textbf{Line 1} Declare the name of the module. In large projects the module name allows SWIG to create wrappers that don't have issues with namespace resolution.
      \item \textbf{Lines 3-5} This is a special block that is copied and pasted, with out SWIG parsing, directly into the generated C/C++ portion of the wrapper. If there are things that need to happen for the underlying source to function, but SWIG doesn't need to know about, they go here.
      \item \textbf{Lines 7-12} Notice the use of the \texttt{\%include} where C++ programmers are used to seeing \texttt{\#include}. This is a special SWIG statement that instructs SWIG to access the file \texttt{"std\_vector.i"} (included as part of SWIG) and give the interface access to the \texttt{vector} class from within the namespace \texttt{std}. I then then expose the typdefs found in \texttt{"common.h"} as swig \texttt{template}s.
      \item \textbf{Line 14} The SWIG \texttt{\%import} directive is used to tell the wrapper that important items live in \texttt{common.h}, but that no wrapper code needs to be generated for that file.
      \item \textbf{Line 15} Finally the SWIG \texttt{\%include} directive is used to include the main file \texttt{HKnotVector.h} in the generated wrapper.
    \end{itemize}
    \mainstretch{}

    \noindent Although the interface file is only 15 lines long, it is fairly complex: it took more than 2 days of reading documentation to produce.\COMMENT{The amount of time that you spent may not be relevant.}  One thing to note about this interface is that when it is run, the entire \texttt{HKnotVector} class (really everything defined in \texttt{HKnotVector.h}) is wrapped and exposed to the target language. This could pose problems if various types, functions, or class attributes shouldn't be accessed outside of C or C++.

    Using this file is a two-step process: 1) Run SWIG on the \texttt{"HKnotVector.i"} and generate the interface, 2) incorporate the generated files into a build system so that they can be \texttt{import}ed into Python. This first step is very straightforward and can be accomplished by running the following from the command line:

    \setstretch{0.7}
    \vspace{.2in}
    \begin{lstlisting}
      swig -c++ -python HKnotVector.i
    \end{lstlisting}
    \mainstretch{}

    \noindent This command runs SWIG, tells it that the source language is C++, the target language is Python and that the interface file is \texttt{HKnotVector.i}. After running the command two files will be generated \texttt{HKnotVector\_wrap.cxx} and \texttt{hsfpy.py}. Together these files files make up the wrapper of HKnotVector.

    The next step is to incorporate these files into a build system so that they can be compiled in a way that the system Python can interact with them. The SWIG documentation gives a few possible methods for doing this, but the recommended solution is to let Python handle the compiling. This will ensure that the correct libraries are linked at compile time and that the version of Python directing the compilation will be able to use the compiled extensions. To do this, a \texttt{setup.py} file must be created. The \texttt{setup.py} file for this example appears below (note that an explanation of key parts of the file is given after the code).

    \setstretch{0.7}
    \vspace{.2in}
    \lstinputlisting[language=python,label=code:SWIGsetup.py, caption=\texttt{setup.py} file for SWIG]{../code/swig/setup.py}
    \setstretch{1.35}

    \begin{itemize}
      \item \textbf{Line 6} From the Python  \texttt{distutils} package, import the \texttt{setup} function and the \texttt{Extension} class. The \texttt{setup} function is the main driving point in this file and will direct the compilation. Objects of type \texttt{Extension} hold all the information the \texttt{setup} function needs to compile the extensions.
      \item \textbf{Lines 8-10} Describe the \texttt{HKnotVector} extension. Notice the first argument given to the \texttt{Extension} constructor is \texttt{"\_hsfpy"}. This argument tells the \texttt{setup} function what to name the shared object (or dynamic linking library on Windows) where the compiled wrapper will be placed. Without custom configuration, SWIG requires that this name be a leading underscore followed by the \texttt{\%module} name defined in the interface file.
      \item \textbf{Lines 12-18} Call the \texttt{setup} function to build all the \texttt{Extension}s in the \texttt{ext\_modules} list. This is also where other metadata about the project goes.
    \end{itemize}
    \mainstretch{}

    \noindent The final step in building the interface is to have python compile the wrappers. This is done on the command line with a single command:

    \setstretch{0.7}
    \vspace{.2in}
    \begin{lstlisting}
      python setup.py build_ext --inplace
    \end{lstlisting}
    \mainstretch{}

    \noindent This command tells the system Python (whatever \texttt{python} resolves to on the user's \texttt{\$PATH}) to build the extensions outlined in \texttt{setup.py} inplace, meaning in the current working directory.

    In the end, I decided not to use SWIG to create the interface to the entire \texttt{hsf} library. The verbose C-like interface files and the need to create a separate interface file for each source file made SWIG more difficult than necessary. In addition, the fact that all code from an exposed C++ source file is wrapped was overkill for this project. Also, not all C++ languages features that appear in \texttt{hsf} are supported by SWIG. However, as can be seen from this small exercise, it is a fairly straightforward, if tedious, process to use SWIG to create a Python interface to C++ code.  \REVIEW{Do you want to mention any of the missing C++ features in SWIG?}

\subsection{Boost.Python} \label{sub:boost_python}
  Boost.Python\footnote{Boost.Python is part of the free peer-reviewed Boost project. Boost can be downloaded from the main projects webstie at \url{http://www.boost.org/}. The documentation for Boost.Python can be found at \url{http://www.boost.org/doc/libs/1_54_0/libs/python/doc/index.html}.} (henceforth Boost for short) is an alternative to SWIG and is a highly specialized tool for wrapping C++ for Python use. This apparent lack of flexibility has allowed the Boost developers to provide a very natural and complete coverage of the C++ language. Some key C++ features that are supported in boost are

  \begin{itemize}
    \itemsep -.1in
    \item References and Pointers
    \item Efficient function overloading
    \item C++ to Python exception translation (cuts down on \texttt{SEGFAULT}s)
    \item Functions or methods with default and keyword arguments
    \item Exporting C++ iterators as Python iterators
    \item Control over Python documentation strings
  \end{itemize}

  On the other hand, Boost has significant limitations. First, Boost has a difficult \texttt{Bjam} utility for compiling the wrappers. \texttt{Bjam} is similar to \texttt{make}, but has a difficult and strange syntax. Second, the generated wrapper code is generally very verbose. While is is probably due to supporting some C++ features that other wrapping tools do not, it has at least two major drawbacks: 1) It takes a long time to compile the wrappers and 2) the Python-side execution is typically noticeably slower than the code generated with other tools. Finally, the major drawback and ultimate reason why I did not use Boost for \texttt{hsfpy} is that it is very difficult to install. After reading the (sparse) documentation and searching the internet, I still could not get Boost.Python correctly installed and configured on my system. This would be a major roadblock to future users of the Python bindings and would actually detract from the main justification for creating the bindings: lowering the bar to entry for using \texttt{hsf} in research. For these reasons, I will not include a usage example for Boost.Python, but because I spent quite a bit of time on it and many people seem to like it, I felt it needed to be addressed in this report.

\subsection{Cython} \label{sub:cython}

  The next tool I examined was \texttt{Cython} \footnote{\texttt{Cython} is free and open source. The source code is hosted at \url{https://github.com/cython/cython} and the homepage for the project is \url{http://cython.org/}. }. Instead of being a tool used solely to wrap compiled languages for use in Python, Cython is actually a super-set of the Python language; anything that is valid Python code is also valid Cython code. However, Cython adds a few major improvements:

  \setstretch{1.35}
  \begin{itemize}
    \item Variables, functions, and classes can be given static types. This avoids much of the overhead inherent in a "duck-typed" interpreted language like Python.
    \item Cython programs can make direct calls to C, C++, and Fortran code. This allows the user to directly mix python with low-level, high performance compiled code.
  \end{itemize}

  \mainstretch{}
  Cython accomplishes this by translating the Cython code directly to C or C++, which can then be compiled and loaded into any Python script or session. This means that blocks of code where all objects have been given static types can be written directly in C and therefore achieve almost\footnote{The almost is necessary because there is small overhead in calling the compiled routines from Python and getting the results back.} C-like performance. In addition, the ability to directly call C, C++, or Fortran makes Cython a viable option for wrapping low-level code for use in Python. In Appendix \ref{app:cython_type_example}, I provide a detailed example static typing in Cython.

  \subsubsection{Cython Wrapping Example} \label{ssub:cython_wrapping_example}

    Building on the static typing example in Appendix \ref{app:cython_type_example}, I will now show how to use Cython to wrap the HKnotVector example\footnote{Readers who are unfamiliar with Cython, especially static typing in Cython, are encouraged to read Appendix \ref{app:cython_type_example} before proceeding.}. In order to use external libraries, you need to tell Cython two things: 1) what file the external components are defined in (usually a header \texttt{.h} file) and 2) which parts of that file you would like to access from Cython. For example, instead of calling \texttt{from libc.math cimport sqrt}, I could have done the following:

    \setstretch{0.7}
    \vspace{.2in}
    \begin{lstlisting}[language=python]
    cdef extern from "math.h":
        double sqrt(double x)
    \end{lstlisting}
    \mainstretch{}

    \noindent In the first line I started a \texttt{cdef extern} block. The syntax is simply \texttt{cdef extern from <headerName>:}, where \texttt{headerName} is the name of the external file where the desired objects are defined (\texttt{ "math.h"} for this example). Everything in the indented block following the \texttt{:} is part of this \texttt{cdef extern} block and contains the external declarations that need to be exposed to Cython.

    % TODO: this paragraph is poorly written. I should probably revise it
    In a larger project, it is often necessary to create a Cython interface file (with a \texttt{.pxd} extension), which does for Cython what a \texttt{.h} interface file does for C/C++ . This is necessary when you have multiple Cython \texttt{.pyx} files that need to access the same external source. The declarations go into a  \texttt{cdef extern} block in a \texttt{.pxd} file. This interface is very similar to the C/C++ interface; often users can copy and paste directly from C to Cython. The actual implementation will go into a file with the same name, but with a \texttt{.pyx} extension. This is very similar to \texttt{.h} and \texttt{.c}  files for C. Furthermore, when wrapping a set of C++ classes, people often put the extern definitions in a file named something like \texttt{cpp\_<headerName>.pxd} and Cython declarations in a file named \texttt{<headerName>.pxd}. This is important because it is generally necessary to have one interface file for external declarations (the file named \texttt{cpp\_}), and another interface file exposing the Cython implementation.

    The structure of a Cython wrapper is best understood by example, which I now show as I wrap HKnotVecter.
    I begin with Listing \ref{code:hkxCPPpxd}, which is the file \texttt{cpp\_HKnotVector.pxd}. In this file I use the \texttt{vector} class defined in \texttt{libcpp.vector} and include the all the declarations that appear in \texttt{HKnotVector.h} (Listing \ref{code:HKnotVector} in Appendix \ref{app:code}).

    \setstretch{0.7}
    \lstinputlisting[language=python, label=code:hkxCPPpxd, caption=\texttt{cpp\_HKnotVector.pxd}]{../code/cython/hsfpy/cpp_HKnotVector.pxd}
    \mainstretch{}

    \noindent The next part of the wrapper is the Cython interface \texttt{HKnotVector.pxd }in Listing \ref{code:hkvpxd}. This is a very minimal file that declares the \texttt{HKnotVector} class and sets up some initial attributes of the class.

    \setstretch{0.7}
    \lstinputlisting[language=python, label=code:hkvpxd, caption=\texttt{HKnotVector.pxd}]{../code/cython/hsfpy/HKnotVector.pxd}
    \mainstretch{}

    The final part and main of the wrapper is \texttt{HKnotVector.pyx}, shown here in Listing \ref{code:hkvpyx}. This is where all attributes and methods declared in either of the interface files are implemented.

    \setstretch{0.7}
    \lstinputlisting[language=python, label=code:hkvpyx, caption=\texttt{HKnotVector.pyx}]{../code/cython/hsfpy/HKnotVector.pyx}
    \setstretch{1.35}

    \begin{itemize}
      \item \textbf{Lines 1-6} All necessary items are imported and set up. Notice that neither of the interface files are not actually imported here. If a \texttt{.pxd} and a \texttt{.pyx} file are in the same directory and have the same name, then all things imported or defined in the \texttt{.pxd} are automatically available in the \texttt{.pyx} file.
      \item \textbf{Lines 9-16} Use \texttt{cdef} to declare the class and set up a few special Cython methods. \texttt{\_\_cinit\_\_} is called immediately after the user tries to create an instance of the class and usually holds the minimial setup required to avoid  a \texttt{SEGFAULT} from null pointers. The \texttt{\_\_dealloc\_\_} method is called when the object is passed through the Python garbage collector and is implemented here to avoid memory leaks.
      \item \textbf{Lines 18-35} Here the overloaded constructor for HKnotVector is set up. Two private methods (private by convention of starting with a single underscore) are implemented to handle each the overloads. The \texttt{\_\_init\_\_} method is called after \texttt{\_\_cinit\_\_} when an \texttt{HKnotVector} instance is created and is implemented to dispatch object creation to one of the overloaded constructors.
      \item \textbf{Lines 37-53} The methods declared in \texttt{cpp\_HKnotVector.pxd} are implemented. Pretty much the only thing that needs to happen here is type checking. To do this I use \texttt{cdef} to statically declare variable types and cast objects using \texttt{< $\cdot$ >}.
    \end{itemize}
    \mainstretch{}

    Now that the wrapper is completed, it needs to be incorporated into a build system and complied into a shared object so that Python can access it. As before, we let python\COMMENT{Make sure your Python case is consistent throughout.} handle this step using a \texttt{setup.py} file, which I have included in Listing \ref{code:cySetup}. There are only a few differences between this file and the other \texttt{setup.py} files presented earlier. First, in lines 6 and 10 I explicitly specify the include directories for the \texttt{HKnotVector} extension. Also, I setup the \texttt{hsfpy} package with a module named \texttt{HKnotVector}. This happens on lines 8 and 17.

    \setstretch{0.7}
    \lstinputlisting[language=python, label=code:cySetup, caption=\texttt{setup.py} for Cython wrapper of HKnotVector]{../code/cython/hkv_setup.py}
    \mainstretch{}

    As can be see from this example, wrapping code using Cython provides absolute control over the structure and feel of the wrapper, but it takes more work than, for example, SWIG. I have only wrapped a very small portion of the \texttt{hsf} library in this example, but it illustrates the point. The sheer size of the \texttt{hsf} library makes it unreasonable to construct a wrapper by hand using Cython. Additionally, the core \texttt{hsf} C++ library is still being developed and is therefore liable to change at any time. Trying to keep the Cython wrapper up to date would be a difficult and error-prone task. For these reasons, I decided not to use a by-hand Cython approach in creating \texttt{hsfpy}.

\subsection{XDress} \label{sub:XDress}

  The final tool I evaluated when creating the Python wrapper for \texttt{hsf} is XDress \footnote{XDress is free and open source. The source code is hosted at \url{https://github.com/xdress/xdress} and the homepage for the project is \url{http://xdress.org/}.}. XDress is a very young project that first appeared on github in April 2013. XDress is written in pure python and is an automatic Python wrapper generator for C and C++ source. It constructs the wrapper in a three stage process.

  \setstretch{1.35}
  \begin{enumerate}
    \item External (to XDress) parsing tools are run on the source and a static xml representation of the data structures is generated. Currently, XDress uses GCC-XML\footnote{GCC-XML is free and open source. The source code is hosted at \url{https://github.com/gccxml/gccxml} and the homepage for the project is \url{http://gccxml.github.io/HTML/Index.html}.} for C++ parsing and pycparser\footnote{pycparser is free and open source. The source code is hosed at \url{https://github.com/eliben/pycparser} and the (limited) documentation is found in README.rst in the source.} for C.
    \item The generated xml files are parsed and the C-based API is described in terms of an internal XDress typesytem. This typesystem is very dynamic and was designed from the ground up with API generation in mind. It is the main enabling feature of XDress.
    \item XDress uses various built-in and/or user-supplied plugins to take the API stored in the typesystem and form Cython bindings.
  \end{enumerate}
  \mainstretch{}

  \subsubsection{\texttt{xdressrc.py}} \label{ssub:xdressrc}


    Compared to the other methods discussed here, XDress is very easy to use . The main point of entry for using XDress is to call \texttt{xdress} from the command line. When this command is executed (with no extra arguments options) it will scan the current directory for a file named \texttt{xdressrc.py}. All the instructions for XDress are put into this single python file. It is easiest to understand the types of instructions that need to be in this file by example, so I present one here.

    \setstretch{0.7}
    \vspace{.2in}
    \lstinputlisting[language=python,label=code:samplexdressrc, caption=Sample \texttt{xdressrc.py} for HKnotVector]{../code/xdress/samplerc.py}
    \setstretch{1.35}

    \begin{itemize}
      \item \textbf{Lines 1-3} Set the name of the Cython package, the output directory where the Cython wrapper will go, and the name of the directory where the C/C++ source lives.
      \item \textbf{Lines 5-7} This is an optional step where the user can specify which plugins should be run when \texttt{xdress} is executed. All but the last plugin (\texttt{'foopack.barplug'}) are built-in plugins that come with XDress.  They perform the following functions:
      \begin{itemize}
        \item \texttt{'xdress.stlwrap'}: Generates wrapper a for C++ STL objects (see next bullet for more information)
        \item \texttt{'xdress.autoall'} and \texttt{'xdress.autodescribe'}: Parse all included files and enter all objects into the typesystem
        \item \texttt{'xdress.cythongen'}: Use the generated typesystem to actually write out the Cython files that define the wrapper
        \item \texttt{'foopack.barplug'}: Run the user supplied plugin \texttt{barplug} found in the package \texttt{foopack}.
      \end{itemize}
      Note that if the \texttt{plugins} list is omitted from this file that XDress will automatically populate this list with the necessary plugins to create the Cython interface.
      \item \textbf{Lines 8-14} Specify which STL containers to create Cython wrappers for. These wrappers will be exposed to Python via custom NumPy dtypes that do all data sharing in memory (no copying, which means a lower memory footprint and faster performance). Notice that the specifications here can take on a nested form to accomidate arbitrary complexity. Also note that non-native C/C++ types can be specified here, with the restriction that the user-defined types be mentioned in the \texttt{classes} list below (see next bullet).
      \item \textbf{Lines 16-19}: An optional list of classes XDress should generate wrappers for. The \texttt{classes} object specified here should be a list of tuples. There are various formats for specifying the contents of each tuple, but the format used here is \texttt{('source\_name', 'source\_file', 'target\_name', 'target\_file')}. Where \texttt{source\_name} is the name of the class in C++, \texttt{source\_file} is the file where the class is defined in C++ and the \texttt{target} variants are the name and file for how and where the class should be defined in C++.
      \item \textbf{Lines 21-23}: Optional lists of functions and variables that should be wrapped. The syntax is similar to the syntax for classes.
    \end{itemize}
    \mainstretch{}

  \subsubsection{XDress Plugins} \label{ssub:xdress_plugins}

    As can be seen, specifying the API elements that need to be wrapped is straightforward and simple. To complement this simplicity XDress has a very easy to use plugin architecture that gives users absolute control over how the wrapping is handled. The plugin system is not a mere afterthought, but build into the core of how XDress operates. All the major functionality of XDress is modularized into distinct plugins that are executed using this archtecture. This means that user-supplied plugins will be given the same precedence as built-in plugins. To demonstrate some of the possibilities for XDress plugins, I will explain two of the plugins I have written to handle issues encountered with wrapping \texttt{hsf}.

    The first of these plugins is now a part of XDress and lives in \texttt{xdress.descfilter}. This plugin allows the user to instruct XDress to "filter out" certain API components from the generated wrapper. This can be done in one of two ways: 1) specify that functions or methods with certain types in the function signature be excluded or 2) specify that certain methods of a class should be excluded. This flexibility can be useful when there are certain functions that shouldn't be exposed to Python. This is also a useful stop-gap for data types that have not yet been implemented into the XDress type system. In Section \ref{sub:xdress_and_hsfpy} I will present the actual \texttt{xdressrc.py} file currently being developed for \texttt{hsf}, which will provide a usage example for \texttt{xdress.descfilter}.

    The second plugin is also part of XDress and lives in  \texttt{xdress.doxygen}. This plugin uses dOxygen\footnote{dOxygen is a common documentation utility for C/C++ projects that gives the user the ability to have specially formatted comments in the source code become stylized documentation elements. dOxygen is free and open source. The code is hosted at \url{https://github.com/doxygen/doxygen} and the homepage for the project is \url{http://www.stack.nl/~dimitri/doxygen/}.} to output an xml version of in-line documentation contained in the C/C++ source. The plugin then parses this xml, automatically generates Python docstrings, and inserts them into the Cython wrappers. These docstrings have many uses such as to provide information on methods, classes, or functions when these objects are inspected at the Python interpreter, or to be used by a tool like Sphinx\footnote{Sphinx is a Python package that can automatically create html or pdf (via latex) documentation using the reStructuredText markup language. Additionally, Sphinx can inspect docstrings and turn them into stylized documentation elements, much like dOxygen.} in conjunction with other content to produce stylized documentation. In Section \ref{sub:hsfpy_usage} I will give a preview of what these docstrings look like for \texttt{hsfpy} inside of IPython.

    Another important item to note is that because I became involved with XDress development at a very early stage and because the \texttt{hsf} library utilizes advanced C++ language features, much of the recent and current development of XDress is being driven by the needs that arise in wrapping \texttt{hsf}. This, together with the ease and freedom XDress provides, caused me to choose XDress as the tool to use in constructing \texttt{hsfpy}. As the \texttt{hsf} project moves forward, the relationship with the lead XDress developer, Anthony Scopatz, and the close integration between \texttt{hsf} and the XDress development cycle will very helpful and should ensure long-term functionality.

%%% Local Variables:
%%% mode: latex
%%% TeX-master: "../myStyle/Capstone"
%%% End:
