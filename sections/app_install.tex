%!TEX root = ../myStyle/Capstone.tex
\section{Python Setup and Installation} \label{app:install}

  As motivated in Section \ref{ssec:motivation}, Python was chosen for the high-level interface to \texttt{hsf}. In addition to the \texttt{hsfpy} specific benefits, a working Python environment brings a complete programming language and world-class scientific libraries to the users fingertips. Because Python and most of its packages are open source, there are numberless ways to set up a scientific Python environment. I will describe a few of them here, give a recommendation for which path to take, and walk the reader through the setup process.

  There are two main ways to get a working scientific python distribution:

  \begin{enumerate}[1)]
    \item Build the entire system from scratch. First, Python itself would be installed and then users decide exactly which 3rd party package to include in their Python environment. This is the preferred method for many Python developers and others who are very comfortable at the command prompt.
    \item Use a scientific Python distribution. Users who choose this option get a Python distribution bundled with at least the core scientific stack.
  \end{enumerate}

  For readers of this report, I recommend the second option. Amongst the many scientific Python distributions, I recommend either Enthought Canopy\footnote{Enthought Canopy the successor to the very popular Enthought Python distribution. The homepage Canopy is \url{https://www.enthought.com/products/canopy/}.} or the Anaconda Python distribution from Continuum Analytics\footnote{The homepage for the Anaconda Python distribution is \url{https://store.continuum.io/cshop/anaconda/}.}. Anaconda is a highly specialized distribution focused on scientific or data-driven programming. It includes over 50 of the most useful scientific packages. Enthought Canopy is a more general distribution that bundles over 100 packages, including the core scientific ones. Both distributions have free and paid variants, but as an academic user the complete version of each is available without cost. It should be noted that both Canopy and Anaconda are easily extensible in the same way a distribution built from scratch is; any other Python package can be installed into any Python distribution. Using a scientific distribution simply eliminates much of the initial setup required to get a functional scientific environment.

  For use with \texttt{hsfpy} I recommend using Canopy. For unknown reasons, portions of \texttt{hsfpy} failed to compile using Anaconda, but did compile with Canopy\footnote{This is probably due to the fact that Anaconda is a more bleeding-edge distribution and the included Python is compiled with a relatively new C compiler. Parts of the \texttt{hsf} C++ library are known not to work with the clang C++ compiler, but function fine when using gcc.}. To install Canopy download the basic (free) version of Canopy from \url{https://www.enthought.com/downloads/} and install. Then, to activate the full version as an academic user, register for an Enthought account using an academic email address here: \url{https://www.enthought.com/products/canopy/academic/}. Enthought will send an email to the given email address with a verification link. Once the account is verified, open Canopy and at the top of the window click \texttt{Login} and input the email address and password that were just set up. The final step in getting Canopy fully installed is to click the box labeled \texttt{Package Manager}. In the window that appears, click the button in the bottom right corner that says it will install all available packages. This completes the Canopy setup.

  XDress also has a non-Python dependency: GCC-XML. To install GCC-XML download the source code from \url{https://github.com/gccxml/gccxml/releases/tag/v0.6.x} and follow the installation instructions from \url{http://gccxml.github.io/HTML/Install.html}.
