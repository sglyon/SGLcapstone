%!TEX root = ../myStyle/Capstone.tex

\section{Results and Discussion} \label{sec:results_and_discussion}

XDress is still actively developed and does not yet support all the features of \texttt{hsf}. As such, \texttt{hsfpy} is still a work in progress. It should be noted, however, that the missing functionality in XDress is relatively minor and should be implemented before the end of August 2013. At that time, it will be very easy to finish \texttt{hsfpy}. In the remainder of this section I will present the current version of \texttt{hsfpy}, including the XDress utilities needed to produce it, as well as show some preliminary usage examples.

\subsection{XDress and hsfpy} \label{sub:xdress_and_hsfpy}

  All instructions needed to create \texttt{hsfpy} are contained in a single \texttt{xdressrc.py} file, which I give below in Listing \ref{code:actualxdressrc}. This is the actual file used in development as of August 5, 2013. After presenting the file I will explain key parts.
\REVIEW{I would remove the unecessary lines that are commented out.  Also add more comments (for the point3d's etc.).}
  \setstretch{0.7}
  \vspace{.2in}
  \lstinputlisting[language=python,label=code:actualxdressrc, caption= Actual \texttt{xdressrc.py} for \texttt{hsfpy}]{../code/actual/xdressrc.py}
  \setstretch{1.35}

  \begin{itemize}
    \item \textbf{Lines 1-2} Add the current working directory to the path so the local plugin \texttt{utils.init\_setup} will run\footnote{I have included the contents of \texttt{utils.init\_setup.py} in Appendix \ref{app:initplug}.}.
    \item \textbf{Lines 4-7} Set the package name, package directory, and source directory
    \item \textbf{Lines 9-12} Set options for the \texttt{utils.init\_setup} plugin written specifically for \texttt{hsfpy}. This plugin will take the information contained in \texttt{xdressrc.py} and automatically create the necessary \texttt{setup.py} file. It will also create an \texttt{\_\_init\_\_.py} file in the package directory\footnote{The \texttt{\_\_init\_\_.py} file is used to manage namespaces. The \texttt{utils.init\_setup} plugin automatically imports all wrapped functions and classes into the main \texttt{hsfpy} namespace.}. Finally, the plugin will automatically run \texttt{python setup.py build\_ext --inplace} if \texttt{run\_setup} is set to \texttt{True}.
    \item \textbf{Lines 15-23} List the plugins XDress should use in creating \texttt{hsfpy}.
    \item \textbf{Lines 25-28} Set the \texttt{skiptypes} parameter used by the plugin \texttt{xdress.descfilter}. Any time an item in this list appears as the type of an argument or return value of a function or method, that method is skipped when the wrapper is generated. For the most part this is necessary for two reasons: 1) Un-implemented features of XDress preclude proper wrapping one of these types or 2) There is no need for the Python interface to deal with these types.
    \item \textbf{Lines 30-34} Set the \texttt{skipmethods} parameter used by the \texttt{xdress.descfilter} plugin. This dict contains class names as keys and a list of method names as values. The method names specified in the lists will be excluded from the generated wrapper of the class the list pertains to.
    \item \textbf{Lines 36-46} Which C++ standard library types to create NumPy dtypes for. Note the inclusion of some non-standard types in lines 43-45.
    \item \textbf{Lines 48-70} Which variables, classes, or functions to create wrappers for. The syntax for these lists was explained in Section \ref{ssub:xdressrc}.
  \end{itemize}
  \mainstretch{}

  The \texttt{setup.py} and \texttt{\_\_init\_\_.py} files generated after running xdress are presented in Listing \ref{code:actualsetup} and Listing \ref{code:actualinit}, respectively.

  \setstretch{0.7}
  \vspace{.2in}
  \lstinputlisting[language=python,label=code:actualsetup, caption= Actual \texttt{setup.py} for \texttt{hsfpy}]{../code/actual/setup.py}

  \vspace{.2in}
  \lstinputlisting[language=python,label=code:actualinit, caption= Actual \texttt{\_\_init\_\_.py} for \texttt{hsfpy}]{../code/actual/__init__.py}
  \mainstretch{}

\subsection{hsfpy Usage} \label{sub:hsfpy_usage}

  Although \texttt{hsfpy} is not entirely finished,  there are still some features I can demonstrate that will highlight its functionality.  The first usage example I will provide shows the interactive documentation capabilities automatically built in to \texttt{hsfpy} via the \texttt{xdress.doxygen} plugin.  As noted earlier, the plugin will have dOxygen scan the C++ source for inline documentation included as comments. These comments are then parsed, formatted, and put into the wrapper as stylized Python docstrings.  The docstrings for functions contain the order and type of all input and output arguments and any brief description provided in the source. Additionally, the docstrings for classes contain the basic description of the class as well as a list of all attributes and methods of the class. The content of these docstrings can be accessed interpreter using the \texttt{help} function, directly as a python \texttt{str} from the \texttt{\_\_doc\_\_} attribute of a function or class, or using the \texttt{?} IPython dynamic introspection operator.  Below I have included the content of a sample IPython session.

  \setstretch{0.7}
  \vspace{.2in}
  \lstinputlisting[language=bash, label=code:actualipython, caption= Sample IPython showing docstrings in \texttt{hsfpy}]{../code/actual/ipy_session.txt}
  \mainstretch{}

  As can be seen in the very simple example above, if the user doesn't know what methods a certain object has or what the types of a particular function's arguments are, they can easily pull up the docstrings and find out. This makes \texttt{hsfpy} much more accessible to outside users who have not been involved in the structure and development of \texttt{hsf}. Unlike  the static, complied C++ library, the Python interface invites users to explore the various data types interactively, hopefully encouraging a much quicker development for projects using \texttt{hsf}.

  In the next example, I will instantiate the basic data structures of HSF and show how they can be queried to give information about the underlying objects. Again \texttt{hsfpy} is an unfinished project, but most main data structures have been wrapped. The components that are missing deal with doing the analysis on those data structures. The main object in HSF is the hierarchical spline forest, which is implemented as \texttt{HForest}. The spline forest is created from a collection of spline trees, represented in the code as \texttt{HNurbsTree}. Each of these spline trees is a collection of individual NURBS, which corresponds to the \texttt{HNurbs} class. Finally, the \texttt{HNurbs} objects are built using C++ \texttt{vector}s of \texttt{HKnotVector} objects, which represent a knot vector. Listing \ref{code:longexample} shows how these objects might be created. Note that in Listing \ref{code:longexample} I print various results and objects to a file; this file appears in Listing \ref{code:longoutput}.
\REVIEW{The second knot vector needs to be obtained by subdividing the previous vector (the first entries would be [0, 0, 0, 0.1, 0.2, ...]).  This is a (currently) an assumption of an hsf HForest.}
  \setstretch{0.7}
  \vspace{.2in}
  \lstinputlisting[language=python, label=code:longexample, caption= Creating \texttt{hsfpy} objects.]{../code/actual/longexample.py}
  \mainstretch{}

  \setstretch{0.7}
  \vspace{.2in}
  \lstinputlisting[language=bash, label=code:longoutput,  numbers=none, caption= Output of data structure example]{../code/actual/longoutput.txt}
  \mainstretch{}

  As can be seen, \texttt{hsfpy} is well on its way to becoming a functional and seamless Python interface to \texttt{hsf}. In addition to wrapping the analysis portions of \texttt{hsf}, there are a few other things that could make \texttt{hsfpy} even better.  XDress allows the insertion of handwritten functions and methods into the wrapper code via sidecar files. A sidecar file is a Python file that sits along the \texttt{.h} and \texttt{.cpp} source and modifies the internal XDress representation of the objects when XDress is executed. I actually used this feature to get the \texttt{HKnotVector} instances to print nicely in the output file from the example above. In addition to aesthetic improvements, the sidecar files can be used to make the wrapped classes behave more like native Python classes. Adding things like Python indexing (through the \texttt{\_\_getitem\_\_} and \texttt{\_\_setitem\_\_} methods) and overloading standard arithmetic operators (through methods like \texttt{\_\_add\_\_}, \texttt{\_\_sub\_\_}, ect.) will allow the \texttt{hsfpy} classes to behave and feel as if they were implemented in Python initially. However, these classes will have the added benefit of C++ execution speeds and precise memory management.

%%% Local Variables:
%%% mode: latex
%%% TeX-master: "../myStyle/Capstone"
%%% End:
