%!TEX root = ../myStyle/Capstone.tex
\section{Introduction} \label{sec:Intro}

  \subsection{Background} \label{ssec:background}

    A physicist is interested in discovering and explaining why things are the way they are. This is usually done by making observations, isolating important variables or factors, and building models. In order to use and solve these models physicists need a way to represent them visually and/or in terms of mathematical functions. Especially in physics, these mathematical functions are differential or difference equations with an associated set of boundary conditions.

    There are many numerical techniques commonly employed to solve boundary value problems. Among them are finite difference methods (FD), finite element methods (FEM), boundary element methods (BEM), and finite volume methods (FVM). A 2-component approach is taken with each of the techniques:

    \setstretch{1.35}
    \begin{enumerate}[1)]
      \item The geometry of the problem is discretized and represented using a mesh.
      \item One of the above methods is applied to this mesh and a solution to the model is computed.
    \end{enumerate}
    \mainstretch

    Each of the discrete solution techniques mentioned above has its own strengths and weaknesses. FD methods are relatively easy to implement, but are restricted to rectilinear geometry\footnote{It is actually possible to systems with more complex geometries, but it is difficult and using another method is suggested}. FEM, BEM, and are all more flexible in how the geometry can be represented, but are typically more difficult to implement. In section \ref{ssub:splines}, I describe where this additional flexibility comes from.

    \subsubsection{Splines} \label{ssub:splines}

      Often the mesh used in FEM, BEM, or FVM is defined using a system of splines. A spline is a smooth, piecewise defined polynomial function that is also smooth where the polynomials pieces come together \cite{judd1998}.
      For numerical methods, the standard type of spline to use to represent the geometry is the non-uniform rational B-spline (NURBS). The definition of a NURBS starts with a non-decreasing and potentially non-uniform vector of knots, which discretize the domain into smaller regions. Polynomial functions are then defined on each of those regions. Next, a set of weights is applied to the basis functions to define rational functions. Finally, a set of coefficients link the rational basis functions to the geometry that is to be described.

      Once the geometry has been described (with a grid, NURBS or otherwise), the next step in solving a boundary value problem is to perform the analysis on the discrete system. One inefficiency with standard FEM, BEM, and FVM methods is that the mathematical constructs used to represent the geometry are different than those used to perform the analysis. The analysis items are given many names, such as triangular patches, square patches, tetrahedral patches, or hex patches. Any of these patches is a transformed version of the geometrical representation (i.e. NURBS). This poses at least three issues: 1) It is computationally costly to move from one representation to another, 2) it requires a significant amount of labor to convert to and from the analysis and design geometries, and 3) conversion between geometries introduces some error.

    \subsubsection{Isogeometric Analysis (IGA)} \label{ssub:isogeometric_analysis_iga}

      To overcome these issues, a new computational approach called isogeometric analysis (IGA) was introduced in 2005 by Hughes et al. \cite{hughes2005}. The main idea behind IGA is to use the exact same basis functions to represent the geometry \textit{and} do the analysis \cite{scott2013}. This simple idea streamlines the interaction between geometric design and rigorous analysis. IGA also provides many other benefits to the design and analysis process. FEM and BEM can use the smooth, high-order basis functions that describe the geometry to perform the computation and analysis, resulting in more accurate results. Also, BEM are usually plagued by geometric error; IGA removes the need to convert between geometries and completely eliminates this error.

      \subsubsection{HSF} \label{ssub:hsfpy}

        Hierarchal spline forests (HSF) are the focus of current research at BYU and bring additional improvements on top of IGA. Groups of NURBS can be organized into nested, hierarchal structures called spline trees.  The spline trees can then be collected as an unstructured, geometrically conforming arrangement called a spline forest. This forest gives IGA a number of enhancements, among which are the following:

    \setstretch{1.35}
    \begin{itemize}
      \item HSF basis functions have compact support and can be made into a partition of unity.
      \item HSF curves can be made $C^{\infty}$ between knots and $C^{p-k}$ at knots (p is the degree of spline, k is multiplicity of knot). In this way the user can control the degree of continuity at knot locations.
      \item Local refinement of basis functions is possible (not generally true of splines).
      \item Solutions to boundary value problems obtained using HSF curves are both accurate and smooth.
      \item Geometric structure of governing PDEs can be incorporated directly into the basis (for example $\nabla\cdot\mathbf{B} = 0$ in EM, or $\nabla\cdot\mathbf{v} = 0$ in incompressible flow).
    \end{itemize}
    \mainstretch{}

    To accompany the theory behind HSF, a C++ library is being developed that implements these concepts (\texttt{hsf} will henceforth refer to the C++ library). C++ was an appropriate language choice for the implementation of \texttt{hsf} for a number of reasons:

    \setstretch{1.35}
    \begin{itemize}
      \item C++ is a statically typed, compiled programming language. This allows code written in C++ to be executed very fast. For the types of problems IGA and the HSF theory are usually applied to, this is an absolute must.
      \item C++ is an object oriented programming language. This programming paradigm allows the ideas behind \texttt{hsf} (NURBS trees and spline forests, ect.) to be expressed in a very natural way.
      \item C++ is a mature and has a great foothold in the scientific community. This means that there are many highly optimized and well-tested libraries available for use in \texttt{hsf}.
      \item Some advanced language features, like templates and method, function, or operator overloading, allow the code be general, but still compiled.
    \end{itemize}
    \mainstretch{}

    On the other hand, there are some shortcomings to choosing C++ as the primary programming language for \texttt{hsf}:

    \setstretch{1.35}
    \begin{itemize}
      \item Because C++ is a compiled language, quite a bit of time and effort was spent creating a robust, cross-platform build system for \texttt{hsf}.
      \item  C++ is a relatively low-level language. While this does mean it can achieve great performance, it also means that the language is difficult to learn. This can be a barrier to entry for people, especially undergraduate students, who would like to contribute to the development of \texttt{hsf}\footnote{This is apparent in that the main developers of the library are all faculty members.}.
      \item Also due to the low-level nature of C++, it tends to be more verbose than other languages. The amount of C++ code required to do a task is often much more than the amount of Matlab or Python code required to do the same thing.
    \end{itemize}
    \mainstretch{}

    The vision for the \texttt{hsf} library is that it will become the most powerful and flexible discretization package for engineering and physics. That it is implemented in C++ gives \texttt{hsf} the potential of being very powerful, but might also limit its user base. For that reason, the research group has decided on building an interface between the core C++ library and a higher-level language.

  \subsection{Motivation} \label{ssec:motivation}

    There are many possible options for a high-level interface to \texttt{hsf}. Among the most common are Matlab, R, Julia and Python. Each of these languages has its respective strengths. Matlab is among the most popular languages for high-level numerical analysis and computation. R is the standard for open-source statistical programming. Julia couples a dynamic typesystem and advanced multiple dispatch paradigm with a powerful just-in-time compiler to achieve excellent performance for numerical programming tasks\footnote{Julia also supports native calls to C/C++ through the \texttt{ccall} method}. Python, in contrast, features a complete, state of the art scientific analysis framework build on top of a fully functional programming language.

    We decided to use Python to build the interface to C++ for a number of reasons. Python has long had a reputation for being a good "glue" language. The core of the most common implementation of Python, CPython, is actually written in C and thus boasts a native Python-C API. In many ways, the environment most similar to Python is Matlab, but Matlab comes with a hefty price-tag. Python is free, open source, and runs on almost all operating systems. Python is known for its very readable syntax. It is not unreasonable to expect a researcher to be introduced to Python in the morning and be writing meaningful programs by the end of the day\footnote{Obviously a mastery of the language will develop over time, but the point of Python being readable and easy to learn stands.}. Additionally, in Python it is relatively easy to employ parallel processing. The package mpi4py exposes any system implementation of the message passing interface (MPI) to Python. \texttt{hsf} is currently using MPI to implement core algorithms in parallel. Being able to use MPI from Python will help increase the rate of development for parallel \texttt{hsf}.

    These virtues of Python language all come together into the ideal programming environment for the high-level C++ interface, which we call \texttt{hsfpy}. The hope is that a robust and functional implementation of \texttt{hsfpy} will assist in the larger goal of \texttt{hsf} becoming the go-to package for discretization by lowering the bar of entry. This will allow more researchers to use \texttt{hsf} to do their analysis and more students who would like to contribute to the development of \texttt{hsf} itself.

    The core C++ library for \texttt{hsf} is still being actively developed, but is at a very mature state. As of August 2, 2013 there are over 18,000 lines of actual code (excluding blank lines and comments)\footnote{This was determined using the \texttt{cloc} utility} in the library. This has provided a very stable base upon which the Python interface has been developed.

%%% Local Variables:
%%% mode: latex
%%% TeX-master: "../myStyle/Capstone"
%%% End:
